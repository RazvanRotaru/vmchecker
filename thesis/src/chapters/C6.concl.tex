\chapter{Conclusion}
\label{chapter:virt-concl}

This project represents a big improvement to the overall performance of \project.
While the system's ability to automatically evaluate and grade an assignment
has not been affected, its performance has been greatly improved, 
especially at times of high workload.


\begin{itemize}
\item Extending the virtualization methods to LXC and KVM has reduced the project's 
dependence on VMware and Vix-API. These are proprietary technologies and 
the pyvix wrapper is quite difficult to maintain. Also, having all the
executors extend a generic class makes adding a new virtualization type 
a very easy task.

\item Redesigning the communication model has facilitated the installation process
and eased the administrator's job when adding a new course to the system. 
Keys no longer need to be generated for each course, the permission settings
that need to be set are in lesser number, and the storer's security has been
greatly improved.

\item The changes we have made to the {\bf queue-manager} and
{\bf storer-daemon} modules reduced the time required to evaluate a large
set of submissions. 
\end{itemize}
