\chapter{Redesigning the Communication Model}
\label{chapter:virt-comm-model}

\section{Overview}
\label{sec:vmc-overview}

The communication between the tester and the storer is currently done
through SSH. Authentication is done through public-private key pairs.
For example, let's take \textbf{course1}. When setting up the storer
and tester for the course, the user course1 is created on both the
storer and the tester. Keys are generated for each of them, and
the public keys are added to each other's authorized_keys file.
As one can see in \labelindexref{Figure}{img:ssh-comm} there are two 
steps to the communication process: Step (1) is triggered by the student
submitting an assignment. The files are saved in the repository, and a bundle
is created containing the tests and scripts to build and run the tests.
The bundle is transferred by the www-data user to the tester using SSH key 1, 
to the designated location in the course's \textit{config} file.

Step (2) is triggered by the evaluation process being completed. 
The results are returned to the storer, by the queue-manager 
process running with course1's permissions, using SSH key 2. 
The \textit{.vmr} files are saved on the storer to the location 
set in the \textit{course-config} file.

\begin{center}
\fig[scale=.5]{src/img/ssh-comm}{img:ssh-comm}{Communication Model}
\end{center}

\section{Authentication and Security}
\label{sec:vmc-auth}

Even though the communication model between storer and tester is reliable,
adding a new course to the system involves a great amount of work, mostly 
because authorization requires generating and deploying SSH keys and setting
the appropriate permissions.

Another important issue is that by the current model, the tester is the one 
that copies the results to the appropriate location on the storer. This is 
a serious security issue. Even though the tester doesn't really host important
information, an attacker that has access to the tester can potentially access 
and destroy information located on the storer machine.

In order to eliminate this flaw, ideally the tester would have no access to the
storer. Instead of the tester pushing the results to the storer, the storer
would pull the results when notified. Even though this notification will be done
through a less secure channel, the storer's integrity cannot be compromised.

Our approach was to introduce another element to the storer, that takes care
of communicating with the tester or testers. This element will run as a separate
process having permissions to access any of the files in any course's repository.
We'll call this process the \textbf{daemon}.

Although it introduces some additional complexity to \labelref{the diagram}{img:new-comm}
having the daemon in the system actually simplifies a large part of the communication.
First of all, course1's storer no longer needs to hold a SSH key, as the only needed
key held by the daemon. Secondly, the tester doesn't require each course to have
it's own user account anymore. Since there is no possibility to get access to the 
storer from the tester, any administrator that needs to set up a testing environment
can simply access it directly.

The newly defined communication model works in the following way:

The submission of an assignment in the web interface copies the files to the
repo (1) and calls the queue_task() method of the daemon (2) which in turn
creates and sends the bundle to the tester (3).
Upon completion of the evaluation process, the tester calls the notify() 
method of the daemon (4) which copies the results from the tester (5)
and places them in the repository (6).

For a secure daemon some restrictions have to be imposed. First of all, the 
queue_task() method must only be called from the storer and the
notify() method must only be called from the tester. Calls from any other
machines should fail. Secondly, when submitting an assignment to the tester,
the daemon must retain the location in the course's repo, where the results will
be saved, and issue a random key. This way, the location on the storer can't be
modified by an attacker with access to the tester.


\begin{center}
\fig[scale=.5]{src/img/new-comm}{img:new-comm}{New Communication Model}
\end{center}


\section{Daemon's Implementation}
\label{sec:vmc-daemon}

Implementing the daemon was a rather simple task. The methods located in 
\textit{submit.py} that were originally called from the web interface when
a user submitted the assignment, were moved to \textit{storer_daemon.py}.
In order to provide access to these methods, python's XMLRPC was used.
The daemon acts as a server and listens on a designated port for requests.
The callers, which may be the storer itself or the tester machine will
simply connect to the storer's port and issue a request.

Providing selective access to the methods was done by extending the
\textbf{SimpleXMLRPCServer} and \textbf{SimpleXMLRPCRequestHandler} classes.
When a method is called, the dispatcher checks if the function's name is 
in the restricted_functions dictionary. If it is, it then checks if the
client IP matches any of the IPs defined in the list. A client that isn't
in the list is not allowed to call that function, so the request fails.


\lstset{caption=SelectiveServer implementation, language=python, label=lst:selective-srv}
\begin{lstlisting}
class SelectiveHandler(SimpleXMLRPCRequestHandler):
    def _dispatch(self, method, params):
        server = self.server
        clientIP, clientPORT = self.client_address
        if method in server.restricted_functions:
            if not clientIP in server.restricted_functions[method]:
                raise Exception('Method not supported for this IP')
        func = None
        try:
            func = server.funcs[method]
        except KeyError:
            raise Exception('method "%s" is not supported' % method)
        if func is not None:
                return func(*params)
        else:
            raise Exception('method "%s" is not supported' % method)            
        
class SelectiveServer(SimpleXMLRPCServer):
    restricted_functions = {}
    def register_function(self, function, name = None, IP = None):
        if not IP is None:
            if name is None:
                name = function.__name__
            if name in self.restricted_functions:
                self.restricted_functions[name].append(IP)
            else:
                self.restricted_functions[name] = [ IP ]
        SimpleXMLRPCServer.register_function(self,function, name)
        
srv = SelectiveServer(("", RPC_PORT), SelectiveHandler)
srv.register_function(queue_task, IP='127.0.0.1')
\end{lstlisting}

Since the daemon needs access to the repositories in order to create the bundle,
it needs to be run with www-data's permissions or under another dedicated account.
This account needs to be given full read-write permission to the repository
by the \textit{vmchecker-init-course} script.



