\chapter{Introduction}
\label{chapter:intro}

\section{Brief Description of vmchecker}
\label{sec:vmc-history}

\subsection{Motivation}
\label{sub-sec:vmc-history-motiv}

For students in Computer Science classes, programming assignments are the main
way in which they gain practical experience regarding certain concepts and 
technologies. These assignments are also an important opportunity to receive 
feedback and might also count for the student's final grade.

Student assignments are generally easy to assess. The process consists of 
downloading the student's submission, running a number of tests to check that 
it compiles and gives the right output and finally reviewing the source code
to make sure that no obvious mistakes were made.

While the whole process is not a difficult one, it quickly becomes a tedious task
when you need to assess more than 20 student assignments. Also, when a large number
of assignments need to be tested, they are generally divided between teaching 
assistants who do perform the tasks individually. Considering each of them might
test the assignments in a different environment, grading fairness and consistency
is eventually lost.

Another issue with the classical approach to grading programming assignment is that
while providing the student with feedback upon the work he has done is one of the
main objectives, that feedback often arrives too late. Homework is usually graded
days or weeks after completion, so what would have been valuable advice during the
development phase, is now too little too late.

Apart from the code-review stage of assessing a student submission, the process is 
fairly easy to automate. Also, while an automated system may not provide feedback 
on the quality of the submitted code, it could provide the student with immediate
information regarding the behaviour of his homework given certain input data.

In order to automate the testing process as much as possible, the \project project 
was initiated by the Computer Science and Engineering department at the 
University POLITEHNICA of Bucharest.

This project aims to improve the performance of the vmchecker automated grading
system by adding support for multiple virtualization types, simplifying its
design and by insuring its scalability.


\subsection{Features}
\label{sub-sec:vmc-history-features}

{\bf vmchecker} was built with regard to the requirements of active courses. This
meant that the system has a number of mandatory features:

\begin{itemize}
\item A submission that potentially crashes the system during testing will not affect
the overall functionality, as it's sandboxed in a virtual machine.
\item The system can automatically detect if the student has sent his homework late
and apply a penalty.
\item Any homework can be rechecked at any time, if the grader has concerns regarding
the functionality of the submission.
\item The system provides immediate feedback regarding the building and testing 
process to the students. Faulty or incomplete submissions may be resubmitted.
\item Access to the source files and the test output files is given to the grader.
\item The web interface and submission repository are located on a different system
from the one doing the actual testing. So even though the testing system might
be unavailable, students can still submit their homework.
\item The system is able to evaluate submissions in any programming language,
and it may even receive an entire virtual machine for testing.
\end{itemize}


\section{The Architecture of vmchecker}
\label{sec:vmc-architecture}

The vmchecker automated grading system consists of two separate subsystems:
\begin{enumerate}
\item \nameref{sub-sec:storer}
\item \nameref{sub-sec:tester}
\end{enumerate}

\begin{center}
\fig[scale=.6]{src/img/simple-structure}{img:simple-structure}{Simple Structure}
\end{center}

The {\bf storer} subsystem runs a web server that allows the user to upload an archive
containing the user's homework assignment. This archive is stored in a repository,
along with the results of the homework's evaluation.

The {\bf tester} runs a {\bf queue manager}, which awaits tasks from the storer and upon
receiving then proceeds to powering on a virtual machine, compiling the submitted
source code, running certain tests and uploading those tests to the storer.


\subsection{The Storer}
\label{sub-sec:storer}

The storer is basically a Linux running machine that runs an Apache web server
in order to present the student with a web-interface so he can upload his
assignment. As well as running the said web server, the storer also hosts a 
repository of containing all of the student submissions, organized by course,
assignment, student name and submission date. This is done in order to prevent
the accidental overwriting of a submission.

\begin{center}
\fig[scale=.4]{src/img/storer-structure}{img:storer-structure}{Storer Structure}
\end{center}

The storer machine may host repositories for multiple courses. The storer
must insure that only the authorized individuals have access to each repository.
This is done by creating a separate Linux user account to each course, and placing
the repository inside the user's home folder.

Access to the machine is provided by SSH and authorization is done by placing
the public keys of the account's users in the authorized_keys file.

The course's storer folder contains a {\it config} file, which lists the parameters
for each assignment (such as deadline, title, virtual machine name), the 
configuration for each available tester (IP or URL, username, queue path)
and the configuration for each virtual machine.

Also in course's storer you may find a \textit{tests/} folder, which contains an archive
with each assignment's tests, several scripts to build and run each assignment,
a database file, and a \textit{repo/} folder.

\begin{center}
\fig[scale=.4]{src/img/repo-structure}{img:repo-structure}{Repo Structure}
\end{center}

The \textit{repo/} folder contains every submission made to this course. It is organized
by assignment, student name and submission date. A symbolic link is made to the 
most recent submission. The {\it submission} folder holds the original archive uploaded
by the student, a \textit{git/} folder with the unzipped contents of the archive and a 
file named \textit{submission-config} which lists the submission's description, and a
{\it results/} folder which holds the output from the testing process.

\subsection{The Tester}
\label{sub-sec:tester}

The tester is a separate Linux running machine which has, just like the storer,
a separate user account for each course. A process called \textit{queue-manager}
waits for file system events in the \textit{queue/} folder. The storer uploads a new
archive to the folder which consists of the student's submission, the tests, and
the necessary scripts to build and run the files. When a new archive is uploaded,
the queue-manager process extracts the contents to a folder and runs the executor.

\begin{center}
\fig[scale=.4]{src/img/tester-structure}{img:tester-structure}{Tester Structure}
\end{center}

The executor is a script that handles the virtual machine specific interactions.
It powers on the virtual machine, it uploads the files, runs the build script and
 the necessary tests. The test results are saved to \textit{.vmr} files.
The executor copies the result files to the bundle folder, shuts down the
virtual machine, then uploads the results to the storer, in the folder designated
by the \textit{submission-config} file.


\subsection{Storer - Tester Interaction}
\label{sub-sec:interaction}

Communication and file transfer between the storer and the tester is done
through the SSH protocol. 

Since the archived bundle is uploaded from the storer by the www-data user (which
runs the apache server) it needs to have permission to read every course's private key. 
Permission is given by using the Posix ACL file permission system.

After the testing is completed, the queue-manager uploads the results to the storer
using the path listed in the submission-config file. The queue-manager is run by the 
course's tester account so it uses by default the account's private key to upload
the files.

\subsection{Installation}
\label{sub-sec:install}

Installing the vmchecker system is a pretty complicated process.
On the storer system, one simply needs to run the \textit{setup.py} script, and then 
the \textit{vmchecker-init-course} script, with the parameter \textit{storer}. The last script
initializes a new storer folder, with a default config file and an empty database.

The next step is generating the public and private keys and set the appropriate 
permissions so that the \textit{www-data} user account can read them. Then add the appropriate 
settings for the assignments, tester and virtual machines to the {\it config} file.
Keep in mind that the tester and virtual machines have not been set up yet.

The web interface is set up by creating a new user account called vmchecker,
setting up a \textit{public_html} folder in its home folder and placing all the necessary files
to run the interface inside. The next step is creating a new website configuration at 
\textit{/etc/apache2/sites-available/}

Finally, you need to edit the global configuration file at \textit{/etc/vmchecker/config.list}
so the course will be listed in the web interface.

To install the tester a few extra steps are needed. vmchecker uses the VMware-VIX
API to interact with the virtual machines. So you first need to install {\bf VMware
server} or {\bf WMware Wokstation} and the VMware-VIX package, then install the pyVix 
wrapper created by Lucian Grijincu.

The next step is to run the \textit{setup.py} script, then the \textit{vmchecker-init-course} script,
with the parameter tester. The script will initialize the folder with a {\it config} file,
and \textit{queue/} and \textit{tmpunzip/} folders.

The queue-manager script is then run. 

\section{Analysis of the Shortcomings of the Former Architecture}
\label{sec:vmc-analysis}

While it is a very useful and powerful tool, vmchecker has its share of drawbacks.
These are caused by the way it was developed - in an incremental way, tackling 
each issue as it came up. 

While rebooting the entire project might be the right course of action, 
backwards compatibility would be lost and the workload would 
prove monumental. We arrive to the conclusion that for the time being it's best
to improve upon the existing model, while focusing on the major areas of concern.

\subsection{The Virtualization Environments}
\label{sub-sec:vmc-analysis-env}

From its inception the only virtualization environment supported by vmchecker 
has been VMware. This aspect leads to several flaws within the architecture.

First of all VMware-VIX API is needed to interact with the virtual machines. 
While there is only official support for the C language, a python wrapper is
used to allow its integration with vmchecker's modules. It's needless to say
that any modification to the API might break existing functionality and 
maintaining the python wrapper becomes an additional concern.

Beside the fact that it's very difficult to install on a Linux operating system
using VMware-VIX API also has some requirements for the hosted virtual machines:
VMware tools must be installed and running inside the guest in order to communicate
with the virtual machine.

Another issue with VMware is the long startup time and the fact that it isn't 
immediately responsive following the restoration of a snapshot.

\subsection{SSH Keys and Permissions}
\label{sub-sec:vmc-analysis-ssh}

An important issue with vmchecker is the ease of setting up the tester and storer
for a new course. At the moment the process is impeded by the fact that communication
is done through the SSH protocol using public key authentication. 

A new pair of keys needs to be generated, for both the storer and the tester account,
then the appropriate permissions need to be set so that www-data has access to 
the keys.


\subsection{Scalability}
\label{sub-sec:vmc-analysis-scalability}

The existing support for a distributed model in vmchecker is limited. While
you may have multiple testers associated with the same storer, an assignment
can only use one of those testers. While this is useful, since many courses
may share the same storer, it also means that at times you'll have machines
that are sitting idle, and machines that have a queue of over 20 submissions
to be tested.

Also, in cases where the tester is a multicore machine, multiple virtual
machines may be run at the same time. However, because of the current architecture 
of vmchecker, that extra performance of the host goes to waste.

In these cases, a dynamical way of distributing the workload between testers is 
needed.

\section{Improving the Existing Model}
\label{sec:vmc-improving}

This project aims to improve the performance of vmchecker. 

All of the concerns listed in the previous section will be tackled as follows:

\begin{itemize}
\item Support for new virtualization platforms will be added, while keeping support
for VMware. This will ensure a smooth transition and the new version of vmchecker
will be backwards compatible.
\item In order to eliminate the need for SSH keys for each account the communication
model will be modified, such that a daemon process running on the storer will
deal with the communication with the tester machines. The daemon process will also
be the one to fetch the results from the tester machine, thus eliminating the need
for the tester to know the storer's public keys.
\item The queue-manager service will be able to provide status updates to the daemon,
so it will be aware of the number and capabilities of all the available virtual
machines. A load balancing service will be provided by the storer daemon, which
will be able to properly place the submitted assignment in the least used queue.
\end{itemize}
