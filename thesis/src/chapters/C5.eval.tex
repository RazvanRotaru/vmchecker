\chapter{Testing and evaluation}
\label{chapter:virt-testing}

In order to gain a good understanding on how the virtualization methods
compare to each other, a few tests have been done.

The results are listed below:

\paragraph {}

Virtualization method: LXC

\begin{tabular}{ l | c }
Total time & Running time \\
4.292s & 0.4249s  \\
4.311s & 0.4363s  \\
4.250s & 0.4277s  \\
4.311s & 0.4406s  \\
\end{tabular}

\paragraph {}


Virtualization method: VMware

\begin{tabular}{ l | c }
Total time & Running time \\
30.404s & 2.1983s  \\
30.056s & 2.1971s  \\
30.190s & 2.1984s  \\
30.173s & 2.1973s  \\
\end{tabular}

\paragraph {}

Virtualization method: KVM

\begin{tabular}{ l | c }
Total time & Running time \\
26.320s & 0.2821s  \\
25.453s & 0.2298s  \\
28.629s & 0.2450s  \\
26.482s & 0.2831s  \\
\end{tabular}


\paragraph { }

There are some interesting observations that can be drawn from the data. First
of all, it is noticeable that LXC is by far the best virtualization method
in terms of evaluation time. It is almost 10 times as fast as VMware.
As profiling the python scripts revealed, a noticeable part of the time
was occupied by the SSH connection, used for transferring the files in and out
of the container. However, when using the Linux native copy command, {\bf cp},
that time was reduced by up to a second. However, SSH is better way to
copy files to and from the container, as it does not require superuser access.

\paragraph { }

Another interesting observation was that KVM actually has a better running
time of the actual program than any other virtualization method,
 although the total evaluation time is close to that of VMware. 
 However, considering the drawbacks of VMware, that makes KVM a powerful
alternative.

\paragraph { } 

The testing was performed on an Intel Core i7-2630QM machine at 2.0 GHz with 8GB
of RAM, running Ubuntu 11.10. The testing bundle was the same in all cases,
consisting in a computationally intensive homework for the Operating Systems course.


